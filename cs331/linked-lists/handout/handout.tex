\documentclass[12pt]{article}
\usepackage[letterpaper,margin=0.5in]{geometry}

\usepackage{titlesec}
\usepackage{xltxtra,fontspec,xunicode}

\usepackage{minted}

\usemintedstyle{trac}

\defaultfontfeatures{Ligatures={TeX}}
\newfontfamily\mytitlefont{Linux Biolinum}
\newfontfamily\mytextfont{Linux Libertine}
%\setmonofont[Scale=0.86]{Inconsolata}

\titleformat{\chapter}{\mytitlefont\huge}{\chaptertitlename\ \thechapter}{20pt}{\Huge}
\titleformat{\section}{\mytitlefont\Large}{\thesection}{1em}{}
\titleformat{\subsection}{\mytitlefont\large\bfseries}{\thesubsection}{1em}{}
\titleformat{\subsubsection}{\mytitlefont\normalsize\bfseries}{\thesubsubsection}{1em}{}
\titleformat{\paragraph}[runin]{\mytextfont\normalsize\bfseries}{\theparagraph}{1em}{}

\newminted{clojure}{linenos,numbersep=4pt}

\begin{document}

\def\title{Immutable Linked Lists Lab}

\hrule

\mytitlefont

\begin{center}
\begin{Large}
CS 331 --- \title
\end{Large}
\end{center}

\mytextfont

\hrule

\vskip 1em

\section{Introduction and Learning Objectives}

In this lab you will create a simple linked list.  Since it is your
first real lab, you will only write a few functions.

Your learning objectives for this lab are:

\def\objectives{
\begin{enumerate}
   \item Use a Clojure record to construct the add, find, and delete functions.
   \item Use Speclj to test your code, and to catch errors in the
     Broken Solutions.
   \item Use git to properly hand in your assignment.
\end{enumerate}
}
\objectives

\section{Getting Started}

Go to your repository and do a \texttt{git pull}.  You will find the
directory \texttt{linked-lists}.

The structure of a Clojure project is somewhat complex, but you will
learn it quickly.  Enter the \texttt{linked-lists} directory, and you
will see the following folders:

\begin{itemize}
\item \texttt{docs} --- contains a file \texttt{uberdoc.html} containing the documentation from the
  comments.  This is available online; follow the link called ``Marginalia Documentation''.
  \emph{Marginalia} is the name of the program used to generate the documentation.
\item \texttt{handout} --- this contains ths file \texttt{handout.tex}, which is the \LaTeX\ source
  for the current document. 
\item \texttt{src/linked\_lists} --- contains the file \texttt{core.clj}, which contains the running
  code for the lab. 
\item \texttt{spec/linked\_lists} --- contains the file \texttt{core\_spec.clj}, which contains the
  test cases for the lab.
\end{itemize}

\section{Given Code}

\subsection{The Linked List Namespace}

At the beginning of \texttt{core.clj}, you will see these lines:

\begin{clojurecode}
(ns linked_lists.core)

(defrecord Cons [car cdr])

(defn insert-at-beginning 
  "Create a new Cons with element `elt` and list `xx`."
  [elt xx]
  (Cons. elt xx))
\end{clojurecode}

The first line declares a \texttt{namespace}.  All the definitions in this file will be active for
the \texttt{linked\_list.core} namespace.  In many languages, this is called a \emph{module}.  This
allows for different programs to declare variables and functions without having to worry that they
are already defined in another namespace.  It's like a family name; the Kim family and the Wang
family family can both have a child name Ramit without having to worry about confusing people.
Well, people will be confused as to why a Korean family and a Chinese family both gave an Indian
name to their child, but within the Kim family, nobody will expect you to be referring to the child
in the Wang family if you call the name Ramit.

There are ways to override the current namespace when necessary; we'll talk about that in a moment.

The second line is our actual record type.  The record is named \texttt{Cons}, and has two accessors 
\texttt{:car} and \texttt{:cdr}.  These names are historical.  The name \texttt{:car} represents the
first part of a pair, in which we will put the data part of our linked list.  The name stands for
``contents of address register,'' which is how the first implementation of \textsc{Lisp} implemented
lists.  Similarly, \texttt{:cdr} represents the second part of a pair, in which we will usually put
a pointer to the remainder of the list.  This is the ``link'' part of the linked list.  The name
stands for ``contents of decrement register.''

Finally, we have a function definition.  The name of the function is \texttt{insert-at-beginning}.
The next line is called a \emph{documentation string}.  Your editor can usually bring up the
documentation string of a function.  In emacs, the key sequence is \texttt{C-c C-d}.  The parameters
of the function are \texttt{elt} and \texttt{xx}.  Finally, the body of the function is
\texttt{(Cons. elt xx)}.  Notice you have to put a period at the end of \texttt{Cons}.  This is
because \texttt{defrecord} is creating a Java class, and the period tells Clojure that instantiating
a Java class is what you intend.

Now: truth in advertising.  The actual file you have will have a bunch of comments interspersed.
Comments begin with a semicolon in Clojure (and in all Lisps).  By convention, block comments begin
with two semicolons, and inline comments begin with one.  Also, the body of the function that you
are given just has \texttt{nil} there.  You can type in the code to the function.

A function written this way is called a \emph{stub function}.  It is good practice to write the stub
functions you think you will need when you first start your program.  Some people raise an exception
instead of returning \texttt{nil}.  The stub functions serve as place-holders, and also enable you
to start writing test cases right away.

\subsection{Test Cases}

The file \texttt{spec/linked\_lists/core\_spec.clj} contains testing code.  Let's look at some of
it.

The file begins with this declaration:

\begin{clojurecode}
(ns linked_lists.core-spec
  (:require [speclj.core :refer :all]
            [linked_lists.core :refer :all])
  (:import [linked_lists.core Cons]))
\end{clojurecode}

It created a new namespace.  The \texttt{require} code tells it that all the functions in the
\texttt{speclj.core} (our testing framework) and \texttt{linked\_lists.core} (our program)
namespaces should be available in the current namespace.  The \texttt{import} line is necessary
since \texttt{Cons} is declared as a Java class, not as a Clojure function.  In this class, you will
usually be given the namespace declarations, so don't worry if this is a bit fuzzy.

The first test describes the behavior of the \texttt{Cons.} record.  The main macro is called
\texttt{describe}, and you should have one of these for every function (or unit of functionality)
within your program.  Inside the \texttt{describe} form will be multiple \texttt{it} clauses that
describe one aspect of the behavior of what you're testing.  Finally, the body of the \texttt{it}
clause contains one or more  \texttt{should} clauses.

\begin{clojurecode}
(describe "The record declaration"
          (it "should create something"
              (should (Cons. 10 20)))

          (it "should have a car"
              (should= 10 (:car (Cons. 10 20))))

          (it "should have a cdr"
              (should= 20 (:cdr (Cons. 10 20))))

          (it "should be chainable"
              (should= 40 (-> (Cons. 10 (Cons. 20 (Cons. 30 40))) :cdr :cdr :cdr))))
\end{clojurecode}

There are many variations of \texttt{should}.  The \texttt{should} version succeeds if the argument
is true.  There is a \texttt{should-not} that succeeds when the argument is false.  The one you will
likely use most is \texttt{should=}, which takes two arguments.  The first is the expected value,
and the second is the code you want tested.  There are other \texttt{should} forms as well; the
Speclj documentation has a complete list.

Here is another test case.

\begin{clojurecode}
(describe "insert-at-beginning"
          (it "creates a cons cell"
              (should-not= nil (insert-at-beginning 10 nil)))

          (it "should work with empty lists"
              (should= (Cons. 10 nil) (insert-at-beginning 10 nil) ))
          
          (it "should work with lists that have data"
              (let [xx (Cons. 10 (Cons. 20 (Cons. 30 nil)))]
                (should= (Cons. 5 xx) (insert-at-beginning 5 xx) ))))
\end{clojurecode}

Notice the third \texttt{it} clause.  It uses a \texttt{let} to create a value we can refer to
again.

Finally, here is a stub test.

\begin{clojurecode}
(describe "insert-at-end"
          (it "never had a test case."
              (should nil))
)
\end{clojurecode}

\section{A diversion...}

You may have noticed another directory, \texttt{test-example}, in your repository.  It contains a
very cut down version of the lab.  The reason it's there is so you can practice compiling and
testing. 

To try it out, go to the \texttt{test-example} directory.  The layout is similar to the original lab.
Take a look at the code to see what we've included.

To run a test, you will type \texttt{lein spec} from inside the \texttt{test-example} project.  The
result should look something like this:

\begin{verbatim}
$ lein spec
....FFF

Failures:

  1) insert-at-beginning creates a cons cell
     Expected: nil
     not to =: nil
     /home/mattox/class/2013-08/cs331/labs/linked-lists/spec/linked_lists/core_spec.clj:27

  2) insert-at-beginning should work with empty lists
     Expected: <#linked_lists.core.Cons{:car 10, :cdr nil}>
          got: nil (using =)
     /home/mattox/class/2013-08/cs331/labs/linked-lists/spec/linked_lists/core_spec.clj:30

  3) insert-at-beginning should work with lists that have data
     Expected: <#linked_lists.core.Cons{:car 5, :cdr #linked_lists.core.Cons{:car 10, :cdr #linked_lists.core.Cons{:car 20, :cdr #linked_lists.core.Cons{:car 30, :cdr nil}}}}>
          got: nil (using =)
     /home/mattox/class/2013-08/cs331/labs/linked-lists/spec/linked_lists/core_spec.clj:34

Finished in 0.02344 seconds
7 examples, 3 failures
\end{verbatim}

The first line is a quick summary of what happened: the dots are passing tests, the ``F's'' are
failed tests.  If there are failures, the details are spelled out.  In this case, there were three
failures.  As you can see, the code was expecting some \texttt{Cons} cells to be created, but we got
\texttt{nil} instead.  Looking at the source code, you will conclude that this is most likely
because the \texttt{insert-at-beginning} function only returns \texttt{nil}.

Fix it.  If you get stuck just replace the \texttt{nil} with 

\begin{clojurecode}
(Cons. elt xx)
\end{clojurecode}

Next, try running \texttt{lein spec} again.  You should now see something like this:

\begin{verbatim}
$ lein spec
.......

Finished in 0.00159 seconds
7 examples, 0 failures
\end{verbatim}

If you understood what was happening here, you can now go back to the real lab, which is contained
in the \texttt{linked-lists} directory.

\section{Your Work}

Your job now is to write functions along with their corresponding tests.  We have left stub
functions along with documentation in the source code.  It is common practice to write the test
cases \emph{first}, before filling in the stub function.  This way the tests can verify for you
immediately when you are done. You don't have to do it that way, though.

The functions you have to write are as follows:

\begin{itemize}
\item \texttt{insert-at-beginning} --- You did this by now.
\item \texttt{insert-at-end} --- Insert at the far end, copying the whole list as you go.
\item \texttt{sorted-insert} --- Insert in the middle, according to the value of the item.
\item \texttt{search} --- Find an element if the list.
\item \texttt{delete} --- Remove an element from the list.
\item \texttt{delete-all} --- Remove all copies of a given element from the list.
\item \texttt{efficient-delete} --- This one is like \texttt{delete}, but if you try to delete
  something that is not there, it will return a pointer to the \emph{original} list.
\end{itemize}

You should check that functions like insert and delete share memory properly.  Consider the
following two lines:

\begin{clojurecode}
  (should= x y)
  (should (identical? x y))
\end{clojurecode}

The first line checks if \texttt{x} and \texttt{y} are mathematically the same.  The second line
checks if \texttt{x} and \texttt{y} refer to the same memory location.

\subsection{Using Git}

Whenever you make significant changes to your code, it is a good idea to commit your work.  To do
this, first use \texttt{git add} to stage the files you changed.  Then use \texttt{git commit -m
  "Your log message here"} (with an appropriate log message) to explain what you did.  Finally, use
\texttt{git push} to send your work to the bitbucket server.  It is important to do this for a few
reasons.  First, this is how you will turn in your work.  Second, it serves as a nice backup
solution in case your local copy gets messed up or goes missing.

Your lab will be graded by a grading script.  The script will allow you to turn in a copy once every
15 minutes.  When it is done grading, it will write a file called \texttt{linked-lists.txt} to a
directory called \texttt{reports} in your repository.  Check it to see if you lost any points.  You
can submit as many times as you want before the deadline.

The grading script usually starts on Wednesday or Thursday after the lab is assigned.  Once it is
turned on, you have seven days to finish.  We will post on Piazza when this happens.  Since this is
the first lab, expect a false start or two as we get the system online.

\end{document}
