\documentclass[12pt]{article}
\usepackage[letterpaper,margin=0.5in]{geometry}

\usepackage{titlesec}
\usepackage{xltxtra,fontspec,xunicode}

\usepackage{minted}

\usemintedstyle{trac}

\defaultfontfeatures{Ligatures={TeX}}
\newfontfamily\mytitlefont{Linux Biolinum}
\newfontfamily\mytextfont{Linux Libertine}
%\setmonofont[Scale=0.86]{Inconsolata}

\titleformat{\chapter}{\mytitlefont\huge}{\chaptertitlename\ \thechapter}{20pt}{\Huge}
\titleformat{\section}{\mytitlefont\Large}{\thesection}{1em}{}
\titleformat{\subsection}{\mytitlefont\large\bfseries}{\thesubsection}{1em}{}
\titleformat{\subsubsection}{\mytitlefont\normalsize\bfseries}{\thesubsubsection}{1em}{}
\titleformat{\paragraph}[runin]{\mytextfont\normalsize\bfseries}{\theparagraph}{1em}{}

\newminted{clojure}{linenos,numbersep=4pt}

\begin{document}

\def\title{Stack and Queue Lab}

\hrule

\mytitlefont

\begin{center}
\begin{Large}
CS 331 --- \title
\end{Large}
\end{center}

\mytextfont

\hrule

\vskip 1em

\section{Introduction and Learning Objectives}

\def\objectives{
\begin{enumerate}
   \item Implement a stack using persistent lists as a back-end.
   \item Use two persistent lists to make an efficient persistent queue.
   \item Use Speclj to test your code, and to catch errors in the Broken Solutions.
\end{enumerate}
}
\objectives

\section{Getting Started}

Go to your repository and do a \texttt{git pull}.  You will find the
directory \texttt{fifolifo}.

In the \texttt{fifolifo} directory you will see the following folders:

\begin{itemize}
\item \texttt{docs} --- contains a file \texttt{uberdoc.html} containing the documentation from the
  comments.  This is available online; follow the link called ``Marginalia Documentation''.
  \emph{Marginalia} is the name of the program used to generate the documentation.
\item \texttt{handout} --- this contains ths file \texttt{handout.tex}, which is the \LaTeX\ source
  for the current document.
\item \texttt{src/fifolifo} --- contains the file \texttt{core.clj}, which contains the
  running code for the lab.
\item \texttt{spec/fifolifo} --- contains the file \texttt{core\_spec.clj}, which
  contains the test cases for the lab.
\end{itemize}

\section{Given Code}

\subsection{The \texttt{fifolifo} Namespace}

At the beginning of \texttt{core.clj}, you will see these lines:

\begin{clojurecode}
(ns fifolifo.core
   (:refer-clojure :exclude [pop peek]))

(defrecord Stack [top size])
(defrecord Queue [front back size])
\end{clojurecode}

One thing that is new in this namespace is that we have the line
\mint{clojure}|(:refer-clojure :exclude [pop peek])|.  This tells the compiler to not use the
built-in versions of these functions, since we intend to replace them with our own.

We will be using the built-in Clojure lists to handle the details of the stacks and the
doubly-ended-queues, but we will also be using our own wrapper records around them.  This will
allow us to test them more easily, and also allow us to keep track of the size without having
to recompute it each time.

\subsection{Test Cases}

The file \texttt{spec/linked\_lists/core\_spec.clj} contains testing code.  It will be similar
to the last one, except you will propbably want to make more describe forms than
the three that are given.  The third one is just an ``autofail'' test so it's clear who didn't even
attempt the lab.  Delete that one.

\section{Your Work}

Your job now is to write functions along with their corresponding tests.  We have left stub
functions along with documentation in the source code.  

The functions you have to write are as follows:

\begin{itemize}
\item \mint{clojure}|make-stack| 
\item \mint{clojure}|stack-size|
\item \mint{clojure}|push| 
\item \mint{clojure}|pop|
\item \mint{clojure}|top|
\item \mint{clojure}|make-queue|
\item \mint{clojure}|queue-size|
\item \mint{clojure}|enqueue|
\item \mint{clojure}|dequeue|
\item \mint{clojure}|peek|
\end{itemize}

The tests will check to be sure memory was shared properly in the push, enqueue, pop, and dequeue functions.

You may want to add your own function to handle the rotation of the back list to the front list.
That is encouraged.  However, be sure not to write explicit test cases against it.  Since it's not
part of the public specification, things will break when your code is combined with the solutions,
and the script will assume that it is your problem, not ours.

\subsection{Using Git}

A repeat from the last lab:
whenever you make significant changes to your code, it is a good idea to commit your work.
One poor student deleted his \texttt{core.clj} file before doing any commits, and had to start over.
If he had committed something, he could have just done \verb|git checkout core.clj| to recover the
file.

The grading script is fully operational now.  Expect these tests to start on Wednesday.

\end{document}

